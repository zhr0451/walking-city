\documentclass[a4paper,12pt,twoside,twocolumn]{book}

%————————————————————————%
% Main package           %
%————————————————————————%
\usepackage[layout=true]{dnd}   % using build in layout
%————————————————————————%
% Main package fixes     %
%————————————————————————%
\setlength{\headheight}{16pt}
%————————————————————————%
% Hyphenation rules      %
%————————————————————————%
\usepackage{hyphenat}
\hyphenation{сло-го-ра-зу-ме-ние}
%————————————————————————%
% Fonts                  %
%————————————————————————%
\usepackage{fontspec}
\usepackage[utf8]{inputenc}
\setmainfont{CMU Serif} %   cyrillic fonts
\setsansfont{CMU Sans Serif}
\setmonofont{CMU Typewriter Text}
\defaultfontfeatures{Ligatures=TeX}
\DeclareEncodingSubset{TS1}{cmr}{0} % отключает TS1 для CMU
\usepackage[english,russian]{babel}
\usepackage{fontspec}
\defaultfontfeatures{Ligatures=TeX} % без TS1
%————————————————————————%
% Grafics                %
%————————————————————————%
\usepackage{graphicx}    % использование изображений в документе
\usepackage{microtype}   % улучшенная типографика
\usepackage{xcolor}      % независимое от драйверов расширение для цвета текста
\usepackage{textcomp}    % дополнительные символы
\graphicspath{{./files}}
%————————————————————————%
% Hyperlinks             %
%————————————————————————%
\usepackage{hyperref}
%————————————————————————%
% Usage guide            %
%———————————————————————————————————————————————————————————————————————————————————————————%
% В выбранном для написания текста классе используются:                                     %
% 1. Section — Они разделят главы на большие куски связанного текста                        %
% 2. Subsection — Разделяет информацию для лучшего восприятия читателем                     %
% 3. Paragraph — Стиль, используемый в оригинальных книгах для выделения глав.              %
% 4. Subparagraph — Более удобный для восприятия стиль параграфов, выделяется табуляцией    %
% 5. Special Sections                                                                       %
% 5.1 \DndFeatHeader для черт, \DndSpellHeader для заклинаний                               %
% 5.2 \DndArea и \DndSubArea предоставляют возможность автоматической нумерации регионов    %
% 6 Также класс предоставляет текстовыделители                                              %
% 6.1 DndReadAloud используется для текста, который должен зачитать мастер                  %
% 6.2 DndComment может использоваться внутри текста                                         %
% 6.3 DndSidebar предполагается использовать в углу страницы, внизу                         %
% 7. Таблицы                                                                                %
% 7.1 DndTable таблицы в стиле официальных рулбуков                                         %
% 8. Монстры и НРС                                                                          %
% 8.1 DndMonster используется для включения статблоков                                      %
%———————————————————————————————————————————————————————————————————————————————————————————%

\title{Ходок-городок}
\author{papercut interactive}
\date{}
\begin{document}
\maketitle
\tableofcontents
\section{Часть I. Авантюрный налёт.}
\subsection{Авантюрный налёт.}
В первой части приключения \textbf{''Ходок-городок''} героям предстоит совершить налёт на ходячий город, проходящий недалеко от их места остановки и скрыться в городских стенах 
огромной шагающей крепости.
\subsection{Используемые дополнительные механики.}
В приключении используются дополнительные *опциональные* правила. Они призваны разнообразить игровой процесс, а также упростить отслеживание прогресса игроков. 
\\В первой части очень важным фактором является грамотный тайм-менеджмент со стороны игроков. Поэтому было введено правило \textbf{кости событий}.
\subsection{Правило кости событий.}
Время внутри настольно-ролевой игры работает отлично от нашего мира, поэтому было принято решение отслеживать прогресс персонажей исходя из мест, которые они успели посетить
 и дел, которые были выполнены. \textbf{Правило ключевых событий} было придумано для синхронизации действий, которые выполняют игроки и реакций мира, на их события. Кампания предлагает взять кость \textbf{к6} и положить его на грань с цифрой \textbf{1}, после этого следовать указаниям, которые вы встретите по ходу игры. Время от времени игра будет просить перевернуть кость на ту или иную грань, которая будет отвечать за определённый исход событий, к которому стремятся авантюристы. Также, игра будет иногда просить выбрать внешний вид комнаты, испытания или последствие исходя из актуального положения кости.
\paragraph{Зачем это использовать?}
Для того, чтобы разнообразить игру. Данная Кампания является песочницей, чтобы события внутри песочницы оставались контекстно-зависимыми, было придумано это правило, чтобы мир
 отвечал на действия игрока, а мастеру было просто отслеживать прогресс персонажей.
\subsection{Начало приключения.}
Для удобного и наиболее классического начала приключения в городе предусмотрена таверна с постоялым двором, на котором могли бы остановиться авантюристы. Возможно у них есть 
общий знакомый, который предложил им эту авантюрную задачу? Начало кампании, по традиции, отводится мастеру.
\subsection{Структура первой части.}
Первая часть кампании представляет собой классическую городскую песочницу. Игрокам представляется несколько вариантов развития событий, завязанных на ключевых объектах. Также 
предоставляется набор событий, которые игроки могут застать в тех или иных локациях, исходя из положения кости \textbf{к6}, как это было описано выше. 
\\Кампания предлагает 3 варианта попадания на ходячий город. Их условно можно разделить на умный, скрытый и силовой. Каждый из подходов обозначает буквально то, что первым 
приходит в голову. Такой метод ветвления был выбран для того, чтобы предоставить разным партиям игроков получить равное удовольствие от прохождения. Умный метод делает упор 
на \textbf{интеллект} и \textbf{мудрость} персонажей. Скрытый метод делает упор на \textbf{харизму} и \textbf{скрытность} персонажей. Силовой метод прохождения делает упор 
на \textbf{силу} и \textbf{ловкость}. Конечно, игроки могут выбрать промежуточный метод прохождения, это полностью вписывается в рамки приключения, так как оно построено по 
формуле песочницы и даёт игрокам максимальную свободу для решения геймплейных задач.
\subsection{Этапы приключения.}
\paragraph{Этап 1}
Поиск информации о местоположении артефактов и возможных токах входа в ходячий город. На этом этапе ходячий город \textbf{находится далеко от города}. 
\paragraph{Этап 2}
Нахождение артефактов и подготовка к операции подъёма на шагоход. На этом этапе ходячий город стремительно \textbf{приближается к городу}.
\paragraph{Этап 3} 
Проведение операции по подъёму на шагоход. Шагоход \textbf{находится наиболее близко к городу}.
\paragraph{Этап 4}
Последствия подъёма и первые минуты в городе. Шагающий город \textbf{отдаляется от стартовой точки}. 
\subsection{Легенда о Ходке-Городке.}
\paragraph{Что знают игроки?}
Данное приключение начинается с того, что некто (в предлагаемом мной варианте некоторый учёный-волшебник) предлагает авантюристам пробраться внутрь некоторого Ходка-Городка, 
который должен вот-вот пройти мимо Эндгарда, в котором его и дожидаются игроки.
\\''\...исходя из известных нам данных, во времена, существовавшие до ''магической чумы'', магия человечества была развита достаточно для того, чтобы поднимать в воздух города. 
Про истоки, причины, и последствия ''Чумы Заклинаний'' написано уже достаточно научных трудов. Далее в данной работе описывается изучение мифологического сказания о Ходке-городке. 
Хотя, на данный момент, сформирован научный консенсус, который склоняется к онтологическому происхождению этого сказания, существуют основания полагать, что некий 
''Ходок-городок'' действительно существовавшее, либо существующее до сих пор явление.
\\ момент наибольшего магического просветления, послушники храма Гонда бросили все свои силы на производство некоторой механико-магической конструкции, которая и была названа 
''Ходок-Городок''. Исходя из имеющихся у нас данных, эта конструкция должна была стать чем-то вроде ковчега, который спас бы своих послушников от воздействия магической чумы. 
Было бы интересно узнать, какой инструмент либо устройство было задействовано для использования в столь грандиозной стройке.
\\В контексте введения также стоит сказать, что некоторая похожая конструкция была несколько раз замечена около города Эндгард на окраине пустынь на территории бывшего 
Незерила. Из чего была построена гипотеза: примерно в середине весны Ходок-городок проходит по окружной траектории, подходя мимо Эндгарда''.
\paragraph{Зачем игрокам туда лезть?}
У разных игроков разные любимые части в НРИ, поэтому было подготовлено несколько хуков, которые должны сработать на разные группы игроков:
\subparagraph{Сердце Гонда}
Игроки будут гоняться за могущественным магическим артефактом в течении кампании. По легенде Гонд подарил этот артефакт своим послушникам для питания этой монструозной 
крепости.
\subparagraph{Социальная драма}
Кажется, что жизнь обычного жителя этого города полна несчастья. Верховные жрецы закрылись на верху, оставив свой народ умирать. Кажется, полный гнева паладин точно найдёт 
причину дойти до самого верха.
\subparagraph{Магическая чума}
Как же так случилось, что боги обрушили столько гнева на смертных? Разве не интересно заглянуть за занавес времен и узнать обо всём со слов очевидцев?
\subsection{Что случилось на самом деле?}
\textbf{Ходок-городок действительно был построен для защиты служителей храма Гонда} от последствий Чумы Заклинаний. Ведения об этом событии пришли одному из главных жрецов. 
В последний момент гномам удалось достроить эту крепость и эвакуироваться, после чего наступила полная изоляция.
\\
\textbf{Жрец не увидел предсказание полностью}. Город готовился к атаке извне, но он не был готов к тому, что обращаться в нежить начнут горожане. Когда тюрьмы забились, 
верховными жрецами было принято решение перекрыть лифты между уровнями города и скидывать вниз заражённых. Так нижние уровни города превратились в бескрайние трущобы, в 
которых остатки населения пытаются выживать.
\\
\textbf{Глава храма взял на себя обязательство поддерживать движение Ходка-Городка, пока угроза не минует}. Но угроза не минует более никогда, под влиянием Магической Чумы, 
в таком уязвлённом состоянии, он впал в религиозное безумие, вечно ведя свой народ к гибели.
\\
\textbf{В таком состоянии город существует последние сотни лет}, никто не может его покинуть, никто не может на него взойти. Жизнь в нём остановилась. Сохраняя в себе множество 
тайн давно ушедших времён.
\subsection{Город Эндгард.}
\subsection{На краю цивилизации.}
\paragraph{Описание города}
Город находится меж двух гор. Благодаря такому расположению в этой местности формируется собственный микроклимат, который защищает его от горячих пустынных ветров. Также, 
благодаря горному расположению, город нависает над пустыней, позволяя расположить весь город на уровне крепостной стены.
\paragraph{Что происходит в городе?}
В честь очередного ''Ковчегова дня'' городские жрецы вывели весь город на **праздник**. Весь город встречает Ходок-Городок. Городские жители думают будто бы сам Гонд присылает 
им столь зрелищное видение, вдохновляющее его послушников на свершения. Весь город заставлен праздничными палатками, уличными лавками торговцев. Люди повсюду носят карнавальные 
костюмы. Акробаты на ходулях держат на себе монструозные конструкции, повторяющие сам проходящий мимо город.
\\ В честь такого праздника, также устраиваются игрища. Говорят, что победитель, зачерпнёт в себя толику силы этого монструозного гиганта (см. далее).
\paragraph{Празднование 'Ковчегова дня'}    % Это нужно пофиксить, очень кривые кавычки
Для описания празднования на улице можно использовать собственное воображение и таблицы \href{https://dnd.su/articles/inventory/75-trinkets/}{безделушек}. У каждой точки 
интереса также есть свои уникальные ивенты связанные с празднованием.
\paragraph{Как защищён город?}
Город защищён \textbf{стражей}, которая использует \textbf{махолёты} для разведки пустыни вокруг города. Это подразделение стражи обустроило себе взлётно-посадочную полосу 
на крепостной стене, основное же войско базируется около \textbf{Замка лорда Эблкроун}. Город хорошо защищён и патрулируется отрядами стражи.
\subsection{Площадь Единения.}
\paragraph{Описание}
Площадь Единения --- главная площадь города Эндгард, она расположена прямо возле городской стены и открывает невероятный вид на бескрайнюю пустыню.
% тут нужно вставить таблицу с точками интереса
\subsection{Таверна ''Дальний Предел''.}
\paragraph{Описание}
Большая двухэтажная таверна с обширным цокольным этажом, на котором расположилась кухня и служебные помещения. Первый этаж занят большим количеством столов, барной стойкой 
и небольшой сценой. Еда на первый этаж попадает по элеватору с цокольного этажа. Второй этаж занят комнатами для постояльцев.
\paragraph{Развитие сюжета и возможные зацепки}
Приключение начинается здесь. Далее будут указаны возможные варианты развития событий. Как уже говорилось выше, глава подразумевает несколько способов попадания в 
Ходок-городок.
\subsection{Гарнизон разведывательного батальона Эндгарда.}
% Тут нужна иллюстрация
\paragraph{Назначение}
Выше упоминалось, что глава имеет несколько путей прохождения. Эта локация — ключевая точка для \textbf{скрытного} прохождения. Здесь персонажи могут попытаться своровать 
махолёты для того, чтобы улететь скрытно (или под вой городской артиллерии).
\paragraph{Описание}
Гарнизон состоит из защищённой кирпичным забором закрытой территории, на которой расположены \textbf{ангары} и \textbf{оружейная мастерская}.
\paragraph{Ангары}
Ангары представляют из себя 2 небольших овальных здания. Один ангар является ремонтным, второй содержит готовые боевые машины.
\paragraph{Ремонтный ангар}
Ремонтный ангар состоит двух отсеков. Для удобства работы их соединяет сквозной конвейер.
\subparagraph{Организация работы ремонтного ангара}
После вылета махолёта, он попадает в ремонтный отсек, в котором проводятся восстановительные и ремонтные работы, после чего, махолет переходит в отдел снабжения, где его 
дозаправляют.
\subparagraph{Ремонтный отсек}
В центре отсека расположена производственная линия, по которой медленно двигаются махолёты. Над конвейером подвешен кран, он используется для спуска махолётов в случае 
сильных повреждений. Вокруг, на таких же кранах, весит несколько летательных аппаратов.
\subparagraph{Внутри отсека}
В этом зале стоит множество стендов с инструментами, разложены различные механические и магические детали махолётов.
\subparagraph{Отсек снабжения}
% Нужна фотография
В этом отсеке махолёты заправляются топливом, пополняют вооружение, после чего их вывозят обратно во второй ангар, в котором махолёты и хранятся.
\subparagraph{Внутри отсека}
В данном отсеке, в защищённой комнате с надписью на дверях: ''Если кто-нибудь ещё раз попробует закурить около или, не дай Гонд, внутри этой комнаты, я ему все курительные 
места оторву. С уважением, главный изобретатель, Брок''. Хранятся бочки с дымным порохом. Помимо них, в этом зале можно найти легковоспламиняемое масло, которое хранится в 
просмоленных бочках в углу помещения.
\subparagraph{Персонажи}
В этой локации чаще всего можно встретить гномов, служителей Гонда. Они занимаются обслуживанием техники, получая плату из казны. Если они заметят авантюристов, то попросят 
покинуть помещение, сами же, они не заинтересованы в прямой конфронтации. Скорее всего они отложат работу и позовут стражу, если обнаружат рядом с собой авантюристов.
\paragraph{Ангар хранения}
Ангар хранения содержит в себе порядка 20 готовых боевых машин. Это большое готическое помещение с высокими потолками. Внутри расположены комнаты отдыха для персонала и 
небольшая комната первой медицинской помощи. Вокруг машин постоянно суетятся различные существа. В отдалении находится небольшой кабинет, вывеска на двери гласит ''Капитан 
разведывательного корпуса стажи Эндгарда''.
\section{Приложение А. Предметы.}
\subsection{Часть I. ''Авантюрный Налёт''}
\end{document}